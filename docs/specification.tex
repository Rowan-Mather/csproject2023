\documentclass{article}
\usepackage{graphicx} % Required for inserting images
\usepackage[margin=1in]{geometry}
\usepackage[table,xcdraw]{xcolor}
\usepackage{hyperref}
\usepackage{enumitem}
\usepackage{parskip}

\bibliographystyle{ieeetr}
%\bibliographystyle{plain}
\newcommand{\mycomment}[1]{}
\newcommand{\light}[1]{\textcolor{gray}{#1}}

\title{Project Specification: Reconstructing Historical Sites with Augmented Reality}
\author{Rowan Mather}
\date{September 2023}

\begin{document}

\maketitle

\section{Outline}
Overcoming the dissonance between a modern day landscape and its historical counterpart can often be challenging if you are less educated on the area or it is in a state of complete ruin. To aid visitors, many popular historical sites provide an immersive series of images presenting our best guess of how it would have looked. Sketches on notice boards are commonplace, but we are increasingly seeing virtual or augmented reality experiences as well.

Accordingly, the direction of this project is to create a mobile application for viewing user-created 3D models layered over large real-world objects, based on their co-ordinate location. I will refer to these models as a 'site'. Although intended for primarily archaeological purposes, a creator will be able to place down any structure in any location across the world and for other users to walk around and inspect it. 

\section{Existing Work}
I have identified some related existing tools, documented below. However, this project has two crucial distinguishing features:
\begin{enumerate}
    \item It is general-purpose. \\
    Most of the aforementioned augmented reality applications are bespoke, made by an external company to suit the exact needs and layout of the site. This project is not as tailored, but usable anywhere in the world for any model. Often smaller locations/projects will not have the resources to develop their own system, but can use this freely instead.
    \item It is GPS location-based. \\
    Augmented reality applications that require the virtual object to be placed in a specific position often use some form of image 'anchor' or 'marker', such as a table. As you move your phone, the app tracks the relative position of the table and keeps the virtual object still. Since this project operates on on a much larger scale, markers are not as suitable. It uses the absolute position instead.
\end{enumerate}

% Please add the following required packages to your document preamble:
% \usepackage[table,xcdraw]{xcolor}
% Beamer presentation requires \usepackage{colortbl} instead of \usepackage[table,xcdraw]{xcolor}
\begin{table}[]
\begin{tabular}{|l|l|l|}
\hline
\rowcolor[HTML]{EFEFEF} 
\textbf{Technology}   & \textbf{Description}                    & \textbf{Issues (for this application)} \\ \hline
Timelooper \cite{timelooper}     & Viewing reproduced historical moments.  & Fixed in place.                        \\ \hline
Google Earth \cite{googleearth}   & Puts 3D models in a world-view.         & Not live, cannot be widely shared.     \\ \hline
Historik \cite{historik}       & Sites tagged with information.          & Mostly still images of sites.          \\ \hline
Arki \cite{arki}           & Models for construction demo on site.   & Marker-based, built-in 3D editor.      \\ \hline
Morpholio Sketch Walk \cite{morpholio} & Construction layout pulled up into 3D.  & Marker-based, local sharing only.      \\ \hline
Historic VR \cite{historicvr}          & 3D dynamic environments.                & Fixed on LCD screens and bespoke.      \\ \hline
RomeMVR \cite{romemvr}              & Viewing 3D reconstructions over Rome.   & Bespoke, limited movement.             \\ \hline
Tamworth Castle Trail \cite{tamworth} & Virtual people/objects explaining site. & Small-scale, marker-based.             \\ \hline
\end{tabular}
\end{table}

\mycomment{
- https://www.timelooper.com/
lets viewers experience various historical moments—George Washington’s second inaugural address, the construction of the Empire State Building, the Great Fire of London.
fixed point
- google earth
virtual only not immersive, import 3d models and view locally
- https://www.historik.com/
mostly stills
- https://www.darfdesign.com/arki.html
- morpholio sketch walk 
anchor based and local collab only
https://morpholioapps.com/trace/
- https://gamma-ar.com/
seems pretty perfect to me....
- https://historicvr.com/index.php
fixed on lcd screens and bespoke
- rome mvr
https://apps.apple.com/us/app/rome-mvr-time-window/id997392001
bespoke, limited walkabout
- tamworth castle
https://www.tamworthcastle.co.uk/ar-trail
anchors on the floor in each room
}

\section{Tools}
To create the main viewing application, I will be using the Unity game engine, and therefore coding in C\#. Unity has excellent handling of graphics, including VR/AR, and a vast library of assets written by users. Using a tool designed for building graphics-heavy mobile apps will streamline the process, even though it's not a game. 

To publicise sites, the models and their metadata will be simply uploaded to a web-server. Sites will be downloaded as needed from the app. Initially, I will use a GitHub repository as the server.

Model files will be stored in the wavefront (.obj) format. I chose this as it's one of the most ubiquitous file types for 3D models \cite{obj1} \cite{obj2}, which aligns with the idea of my project being a generalist tool. Most 3D modelling software can will import/export it, but I will be using Blender for testing purposes, as well as for any built-in 3D graphics. 

I will be using Unity's location services for determining the GPS co-ordinates of a user. I considered other ways of locating a model, including marker-based \cite{anchor}, and Bluetooth beacons \cite{beacons}. However, due to the larger scale, small budget and requirements to be generally accessible, GPS was the most appropriate single method. 

I will be testing on my personal Android mobile device first and foremost. 

\section{Requirements Analysis and Objectives}
Detailed below are the exact aims for the project with three levels of priority. Any non-core functionality is greyed-out and extension features are further labelled with an 'x'.

I initially discussed this project with an academic who had previously worked on small-scale 3D reconstructions, a secondary school history teacher and an archaeologist. Beyond the immediate use of the main visualisation itself, I asked what other features could be helpful. All the section (D) extension objectives are taken directly from this list.

\begin{enumerate}[label=(\Alph*)]
\item \textbf{User may upload a model to a remote server.}
    \begin{enumerate}[label=\arabic*.]
    \item Models will be in the wavefront (.obj) form.
    \item \light{x. Models can be uploaded in additional common formats e.g. .fbx, .dae, .blend}
    \item User may upload a texture and other metadata to pair with the model.
    \item \light{A subset of the server models can be externally selected to be available in app.}
    \item \light{x. Users cannot upload without submitting a review request.}
    \end{enumerate}
    
\item \textbf{User may download models from the server into their local app.}
    \begin{enumerate}[label=\arabic*.]
    \item Models can be imported from the server in their pure object form.
    \item \light{They can be coloured/textured according to an accompanying texture file (.mtl).}
    \item \light{Users can select a subset of the server models to import locally.}
    \item \light{Models will render with their textures and metadata.}
    \item Models will render with reasonable latency, generally a maximum of 2 seconds per model.
    \end{enumerate}

\item \textbf{User will be able to set their location and move around the models.}
    \begin{enumerate}[label=\arabic*.]
    \item User has a manually configurable geo-location (latitude \& longitude).
    \item User has a manually configurable orientation.
    \item Models can be placed in a geo-location relative to the user.
    \item \light{Models will render only when within a certain distance to the user.}
    \item \light{User can change the render distance.}
    \item \light{x. User can change the scaling of the whole scene.}
    \item \light{User location can be set to their actual mobile device's geo-location.}
    \item \light{User can move freely while the app tracks their geo-location.}
    \item \light{Compass direction can be set to that of device.}
    \item \light{Tilt can be set to the spirit level of device.}
    \item \light{User can switch between manual and device modes.}
    \item \light{Add camera input as background for the scene.}
    \item \light{x. Have a slider for model transparency.}
    \end{enumerate}
    
\item \textbf{Additional information about the sites will be accessible.}
    \begin{enumerate}[label=\arabic*.]
    \item \light{x. Models can be tagged with information in a specific relative location to the model.}
    \item \light{x. Tags can be clicked to view further information/photos.}
    \item \light{x. Tags can be shown/hidden in scene.}
    \item \light{x. A date (range) will be in the corner corresponding to the age of the site you are viewing.}
    \item \light{x. A timeline scroll will be at the bottom of the screen.}
    \item \light{x. Models can be attached to a specific time period and will only be visible when this is selected.}
    \end{enumerate}

\item \textbf{The app will be intuitive and guided.}
    \begin{enumerate}[label=\arabic*.]
    \item \light{x. Include a menu with an app tutorial.}
    \item \light{Most information can be hidden so the user is not overloaded.}
    \item The default view will be a simple drop-down list of available models and device mode.
    \end{enumerate}
\end{enumerate}

\section{Timeline and Methodology}
I have scheduled time for all the core functionality of my project (and some extension/expansion objectives) using a custom Gantt chart (see Figure \ref{fig:gantt}). Since it also functions as a form of checklist, I will be modifying it continuously with updates on my progress. This will allow me to distribute work evenly and recall quickly how far along a section is.

Due to the quite distinct stages, work will be completed following a methodology most akin to incremental \cite{agile}, but with more agile elements. The specification at the start of each iteration will not be fully detailed since I will need to experiment with different tools within Unity, but the overall objectives will be clear. 

I will keep a progress log detailing the time and activity of each work session. This will help track the issues I have faced, and allow me to reschedule tasks where necessary, as well as reporting the successes. 

\newpage
\section{Risks}
\begin{itemize}
    \item Laptop failure. \\
    Since I will be working primarily on my personal laptop, which has Unity installed, it failing would be a set-back. I will keep copies of all the code and documentation on GitHub and Google Docs to mitigate the loss and be able to return to previous successful versions of my work.
    \item Mobile device failure. \\
    If my Android device were to fail, I will be able to continue testing locally on my laptop until I source a new one.
    \item GPS inaccuracy. \\
    GPS with maximal accuracy is reserved for state/military operations, so the location data I will be able to access may be several metres off. Generally, this should not be a problem on a large scale, but I may need to mitigate it with extra manual user input. 
\end{itemize}

\section{Legal and Ethical Considerations}
\begin{itemize}
    \item Unity Terms and Charges \\
    I will need to adhere to the Unity Ts\&Cs as well as those for each asset in the library. Generally for a small, non-commercial product this shouldn't be too great a consideration. 
    \item Model Security \\
    If models are easily up-loadable, there is potential for flooding the server, or for unwanted content to be posted. In addition, a user may not wish to share their material globally instantly. For prototyping purposes, this is a small concern, but to extend the project, there should be restrictions on uploading and sharing, perhaps linked to an account system.
\end{itemize}

\begin{figure}
    \includegraphics[width=600pt, angle=90]{Gantt Chart - Sheet1.pdf}
        \caption{Year Gantt Chart}
        \label{fig:gantt}
\end{figure}

\newpage
\bibliography{bibliography}

\end{document}
